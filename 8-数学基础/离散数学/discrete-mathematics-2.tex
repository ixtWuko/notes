\section{归纳与递归}
\subsection{归纳}
\begin{itemize}
    \item 数学归纳法:(1)$P(1)$为真;(2)假设$P(k)$为真,可以推得$P(k+1)$为真。由两步可以证得$\forall k P(k)$成立。
    \item 强归纳法:(1)$P(1)$为真;(2)假设$P(1), P(2), \cdots, P(k)$全为真,可以推得$P(k+1)$为真。由两步可以证得$\forall k P(k)$成立。也即,证明$P(k+1)$为真不仅需要前一项,而需要之前的所有项。
    \item 尽量使用强归纳法,强归纳法是完全的归纳法。
    \item 数学归纳法和强归纳法的有效性都来自于良序性公理:任何一个非空的非负整数集合都有最小元素。
    \item 奇数个馅饼问题:奇数个人站在一个院子里,彼此间距离不同,他们同时将一个馅饼扔向离自己最近的人,至少有一个人没有被扔到。(使用上述的良序性公理可证)
    \item 具有$n$条边的简单多边形可以三角化成$n-2$个三角形,其中$n > 3$.
    \item 结构归纳法:(1)对于递归定义中的基础步骤所规定的所有元素,结论成立;(2)对于递归步骤中,假设用于构造新元素的所有元素结论成立,则对于新元素结论成立。由以上两步可以证得结论成立。
\end{itemize}

\subsection{递归}
利用递归来定义、利用递归来求解。

\subsection{程序正确性}
证明一个程序的正确性分为两个部分:(1)若程序终止,则必定获得正确的答案;(2)程序总是终止。前一部分称为程序的部分正确性。

若每当对程序段$S$来说,初始断言$p$为真时,其终结断言$q$必为真,则称程序段$S$相对于$p$和$q$部分正确,以$p \{ S \} q$表示程序部分正确。

\section{离散概率}
概率的内容看概率论与数理统计部分。

随机算法可以大致分为两类:(不是遍历的算法,也即随机采样而不能覆盖所有情况。)
\begin{itemize}
    \item 蒙特卡罗算法:采样越多,越近似最优解。尽量找好的,但是不能保找到的是最好的。
    \item 拉斯维加斯算法:采样越多,越有机会找到最优解。尽量找最好的,但是不能保证找到。
\end{itemize}

\section{计数}
\begin{itemize}
    \item 基本计数原则:乘积原则和求和原则。乘积原则是指完成一个任务的多个步骤,分别有多种不同的方法,则完成此任务的方法为各步骤中方法数量的乘积。求和原则是指某个事物存在多种情况,情况之间需要求和。

    \item 容斥原理:求和时将某些情况重复计数,需要减去重复的。

    \item 鸽巢原理/抽屉原理:如果$k+1$个或更多的物体放入$k$个盒子,则至少有一个盒子包含2个或更多的物体。

    广义鸽巢原理:如果$N$个物体放入$k$个盒子,则至少有一个盒子包含了至少$\lceil N/k \rceil$个物体。

    \item 具有$n$个不同元素的集合的$r$排列数是\[P(n,r) = n(n-1)(n-2) \cdots (n-r+1) = \frac {n!}{(n-r)!}\].
    
    \item 具有$n$个不同元素的集合的$r$组合数是\[C(n,r) = \frac {n!}{r!(n-r)!} = \binom{n}{r}\].

    \begin{itemize}
        \item $\displaystyle \binom{n}{r} = \binom{n}{n-r}$
        \item 帕斯卡恒等式:$\displaystyle \binom{n+1}{r} = \binom{n}{r-1} + \binom{n}{r}$
        \item 范德蒙德恒等式:$\displaystyle \binom{m+n}{r} = \sum_{k=0}^r \binom{m}{r-k} \binom{n}{k}$;令$m = r = n$,则$\displaystyle \binom{2n}{n} = \sum_{k=0}^n \binom{n}{k}^2$.
        \item $\displaystyle \binom{n+1}{r+1} = \sum_{j=r}^n \binom{j}{r}$
    \end{itemize}

    \item 二项式定理:\[(x+y)^n = \sum_{r = 0}^{n} \binom{n}{r} x^{n-r} y^r\]

    \begin{itemize}
        \item 令$x = y = 1$,则$\displaystyle \sum_{r=0}^n \binom{n}{r} = 2^n$.
        \item 令$x =-1, y = 1$,则$\displaystyle \sum_{r=0}^n (-1)^r \binom{n}{r} = 0$.
        \item 令$x = 2, y = 1$,则$\displaystyle \sum_{r=0}^n \binom{n}{r} 2^r= 3^n$.
    \end{itemize}

    \item 具有$n$个物体的集合允许重复的$r$排列数是$n^r$.

    具有$n$个物体的集合允许重复的$r$组合数是$\displaystyle \binom{n+r-1}{r} = \binom{n+r-1}{n-1}$.

    \item 设类型1的相同的物体有$n_1$个,类型2的相同的物体有$n_2$个,\ldots,类型$k$的相同的物体有$n_k$个,那么$n$个物体的不同排列数是$\dfrac {n!}{n_1! n_2! \cdots n_k}$.

    \item 把$n$个不同的物体分配到$k$个不同的盒子使得$n_i$个物体放入盒子$i$的方式数等于$\dfrac {n!}{n_1! n_2! \cdots n_k}$.
\end{itemize}

\section{高级计数技术}
\subsection{递推关系}
序列的递推关系是一个等式,用前面的一项或多项来描述第$n$项$a_n$。满足递推关系的一个序列称为该递推关系的解。

\subsubsection*{常系数线性齐次递推关系}
常系数的$k$阶线性齐次递推关系:
\[a_n = c_1 a_{n-1} + c_2 a_{n-2} + \cdots + c_k a_{n-k}\]

其中$c_1,c_2,\cdots,c_k$为实数,$c_k \neq 0$。

假设$a_n = r^n$,则上式可以写成
\[r^n = c_1 r^{n-1} + c_2 r^{n-2} + \cdots + c_k r^{n-k}\]

两侧同除以$r^{n-k}$,得
\[r^k - c_1 r^{k-1} - c_2 r^{k-2} - \cdots - c_{k-1} r - c_k = 0\]

上式称为递推关系的特征方程,此时需要求解特征根$r$.
\begin{itemize}
    \item 若有$k$个不同的特征根$r_1, r_2, \cdots, r_k$,解为
    \[a_n = \alpha_1 r_1^n + \alpha_2 r_2^n + \cdots + \alpha_k r_k^n\]
    其中$\alpha_1, \alpha_2, \cdots, \alpha_k$为常数,使用初始条件解出$\alpha_1, \alpha_2, \cdots, \alpha_k$.
    \item 若有$t$个不同的特征根$r_1, r_2, \cdots, r_t$,其重数分别为$m_1, m_2, \cdots, m_t$,解为
    \begin{align*}
    a_n =
    & (\alpha_{1,0} + \alpha_{1,1} n + \cdots + \alpha_{1,m_1 -1} n ^{m_1-1}) r_1^n \\
    &+ (\alpha_{2,0} + \alpha_{2,1} n + \cdots + \alpha_{2,m_2 -1} n ^{m_2-1}) r_2^n \\
    &+ \cdots + (\alpha_{t,0} + \alpha_{t,1} n + \cdots + \alpha_{t,m_t -1} n ^{m_t-1}) r_t^n
    \end{align*}
    其中所有的$\alpha$为常数,由初始条件解出。
\end{itemize}

以二阶为例,特征方程为$r^2 - c_1 r - c_2 = 0$,解得特征根:
\begin{itemize}
    \item 若$r_1 \neq r_2$,则$a_n = \alpha_1 r_1^n + \alpha_2 r_2^n$.
    \item 若$r_1 = r_2 = r_0$,则$a_n = (\alpha_1 + \alpha_2 n) r_0^n$.
\end{itemize}

\subsubsection*{常系数线性非齐次递推关系}
常系数的$k$阶线性非齐次递推关系:
\[a_n = c_1 a_{n-1} + c_2 a_{n-2} + \cdots + c_k a_{n-k} + F(n)\]

其中$c_1,c_2,\cdots,c_k$为实数,$c_k \neq 0$.

\[F(n) = (b_t n^t + b_{t-1} n^{t-1} + \cdots + b_1 n + b_0)s^n\]

其中$b_0, b_1, \cdots, b_t,s$为实数。

其解的结构是特解$a_n^{(p)}$+通解$a_n^{(h)}$,其中通解为对应的齐次递推关系的解。

下面求其特解:
\begin{itemize}
    \item 若$s$不是特征根,则$a_n^{(p)} = (p_1 n^t + p_{t-1} n^{t-1} + \cdots + p_1 n + p_0) s^n$。
    \item 若$s$是一个$m$重特征根,则$a_n^{(p)} = n^m (p_1 n^t + p_{t-1} n^{t-1} + \cdots + p_1 n + p_0) s^n$.
\end{itemize}

将以上的特解形式带入递推关系,化简并归并同类项,根据不同次的项的系数对应相同,解得以上的各$p$.

\subsubsection*{分治递推关系}
分治递推关系:
\[f(n) = af(n/b) + g(n)\]
\begin{itemize}
    \item 设$f(n) = af(n/b) + c$,且为增函数,其中$b$整除$n$,$a \ge 1,b \ge 1,b \in \mathbb Z,c > 0$。则若$a = 1$,$f(n)$是$O(\log n)$;若$a > 1$,$f(n)$是$O(n^{\log_b a})$.

    进一步,当$n = b^k, k > 0,k \in \mathbb Z$,$f(n) = C_1 n^{\log_b a} + C_2$,其中$C_1 = f(1) + \dfrac c{a-1}, C_2 = -\dfrac c{a-1}$.

    \item \uline{主定理} 设$f(n) = af(n/b) + cn^d$,且为增函数,其中$n = b^k$,$a \ge 1,b \ge 1,b \in \mathbb Z$,$c > 0,d \ge 0$。则若$a < b^d$,$f(n)$是$O(n^d)$;若$a = b^d$,$f(n)$是$O(n^d \log n)$;若$a > b^d$,$f(n)$是$O(n^{\log_b a})$.
\end{itemize}

上述内容可以用于估算含有递归的程序的运行时间,尤其是主定理或更广泛的主定理。

\subsection{生成函数}
实数序列$\{a_0, a_1, \cdots, a_k, \cdots \}$的生成函数是无穷级数$G(x) = \displaystyle \sum_{k=0}^{\infty} a_k x^k$,生成函数的各项的系数为序列的元素。

在计数问题中,通常考虑使用幂级数,下面是一些有用结论:
\begin{itemize}
    \item 设$f(x) = \displaystyle \sum_{k=0}^\infty a_k x^k, g(x) = \displaystyle \sum_{k=0}^\infty b_k x^k$,则
    \[f(x) + g(x) = \sum_{k=0}^\infty (a_k + b_k) x^k\]
    \[f(x)g(x) = \sum_{k=0}^\infty (\sum_{j=0}^k a_j b_{k-j}) x^k\]
    \item 广义二项式系数:
    \[\binom uk= 
    \begin{cases} \displaystyle \frac {u(u-1)(u-2) \cdots (u-k+1)}{k!} & k > 0 \\
    1 & k = 0
    \end{cases}\]
    其中$u$为实数,$k$为非负整数。
    \item 广义二项式定理:
    \[(1+x)^u = \sum_{k = 0}^\infty \binom uk x^k\]
    其中$\lvert x \rvert < 1$,$u$为实数。
\end{itemize}

使用生成函数解决计数问题的例题:

将8块饼干分给3个孩子,每个孩子不少于2块不多于4块,问有多少种分法?

分析:需要寻找3个整数,其和为8。以幂级数考虑,可以使用多项式的乘积,其指数为加和。

解:使用多项式$x^2 + x^3 + x^4$来表示每个孩子可能分得的饼干数量,而3个孩子分得饼干的数量总和可以使用生成函数$(x^2 + x^3 + x^4)^3$来获得。其中,$x^8$的系数即为本题的解。\\

使用生成函数解决递推关系的例题:

求解递推关系$a_k = 3 a_{k-1}$,初始条件$a_0 = 2$.

解:设序列$\{a_n\}$的生成函数$G(x) = \displaystyle \sum _{k=0}^\infty a_k x^k$,
\[xG(x) = \sum _{k=0}^\infty a_k x^{k+1} = \sum _{k=1}^\infty a_{k-1} x^k\]
\begin{align*}
    G(x) - 3xG(x) =& \sum _{k=0}^\infty a_k x^k - 3\sum _{k=1}^\infty a_{k-1} x^k\\
    =& a_0 + \sum _{k=1}^\infty (a_k - 3 a_{k-1}) x^k \\
    =& 2
\end{align*}
\[G(x) = \frac {2}{1-3x} = \sum _{k=0}^\infty 2 \times 3^k x^k\]
所以,$a_n = 2 \times 3^n$.

\subsection{容斥}
\uline {容斥原理}:设$A_1, A_2, \cdots, A_n$是有穷集,则
\begin{multline*}
    \lvert A_1 \cup A_2 \cup \cdots \cup A_n \rvert = \sum _{1 \le i \le n} \lvert A_i \rvert - \sum _{1 \le i < j \le n} \lvert A_i \cap A_j \rvert \\
    + \sum _{1\le i < j < k \le n} \lvert A_i \cap A_j \cap A_k \rvert - \cdots + (-1)^{n+1} \lvert A_1 \cap A_2 \cap \cdots \cap A_n \rvert
\end{multline*}

容斥原理的另一种表述:设$A_i$表示集合中含有性质$P_i$的元素构成的子集,使用$N(P_1, P_2, \cdots, P_k)$表示具有$P_1, P_2, \cdots, P_k$所有这些性质的元素的数量。有
\[\lvert A_1 \cap A_2 \cap \cdots \cap A_k \rvert = N(P_1, P_2, \cdots, P_k)\]

不具有$P_1, P_2, \cdots, P_k$中任何一条性质的元素的数量用$N(P_1', P_2', \cdots, P_k')$表示。

由容斥原理是,可得

\begin{align*}
    N(P_1', P_2', \cdots, P_n') =& N - \lvert A_1 \cup A_2 \cup \cdots \cup A_n \rvert \\
    =& N - \sum _{1 \le i \le n} N(P_i) + \sum _{1 \le i < j \le k} N(P_i P_j) - \sum _{1 \le i < j < k \le n} N(P_i P_j P_k)\\
    &+ \cdots + (-1)^n N(P_1 P_2 \cdots P_n)
\end{align*}

\subsubsection*{容斥原理的应用}
\begin{itemize}
    \item 埃拉托色妮筛:寻找不超过一个给定正整数的素数。一次去掉被2整除的数,被3整除的数,被5整除的数,\ldots

    求这些素数的个数,可以使用容斥原理。

    \item 求两个集合间的映上函数的数量:$m, n \in \mathbb Z^+ ,m \ge n$,则存在$n^m - C(n ,1)(n-1)^m + C(n,2)(n-2)^m - \cdots + (-1)^{n-1} C(n, n-1) \times 1^m \cdot 1^m$个从$m$元素集合到$n$元素集合的映上函数。

    \item 错位排列指没有一个元素处于初始位置的排列。

    $n$元素集合的错位排列数是$D_n = n! \bigg[ 1- \cfrac 1{1!} + \dfrac 1{2!} - \dfrac 1{3!} + \cdots + (-1)^n \dfrac 1{n!} \bigg]$.
\end{itemize}