\section{逻辑、证明}

\subsection{命题与逻辑}
\begin{itemize}
    \item 什么是命题? 或真或假的陈述语句。
    \item $\lnot p: \mathrm{NOT} \ p$ \quad $p \land q:p \ \mathrm{AND} \ q$ \quad $p \lor q:p \ \mathrm{OR} \ q$ \quad $p \oplus q:p \ \mathrm{XOR} \ q$
    \item Condition Statement (条件语句) $p \rightarrow q$: if p is true than q is true.
    \item Bicondition Statement $p \leftrightarrow q$.当$p \leftrightarrow q$恒真,则$p,q$等价($\equiv$)。
    \item 永真式(永远为真的复合命题)、矛盾式(永远为假的复合命题)、可能式
    \item 命题函数:设命题$P(x)$为一个关于变量$x$的函数,则称$P(x)$为命题函数,其中$P$称为谓语。
    \item 全称量词$\forall$、存在量词$\exists$
\end{itemize}

\subsubsection*{常用逻辑等式}
\begin{itemize}
    \item $p \land T \equiv p$ \quad $p \lor F \equiv p$ \quad $p \lor T \equiv T$ \quad $p \land F \equiv F$
    \item $p \lor p \equiv p$ \quad $p \land p \equiv p$ \quad $\lnot (\lnot p) \equiv p$
    \item $p \lor q \equiv q \lor p$ \quad $p \land q \equiv q \land p$
    \item $(p \lor q) \lor r \equiv p \lor (q \lor r)$ \quad $(p \land q) \land r \equiv p \land (q \land r)$
    \item $p \lor (q \land r) \equiv (p \lor q) \land (p \lor r)$ \quad $p \land (q \lor r) \equiv (p \land q) \lor (p \land r)$
    \item $\lnot (p \land q) \equiv \lnot p \lor \lnot q$ \quad $\lnot (p \lor q) \equiv \lnot p \land \lnot q$
    \item $p \lor (p \land q) \equiv p$ \quad $p \land (p \lor q) \equiv p$
    \item $p \lor \lnot p \equiv T$ \quad $p \land \lnot p \equiv F$
    \item $p \rightarrow q \equiv \lnot q \rightarrow \lnot p$
    \item $p \leftrightarrow q \equiv (p \rightarrow q) \land (q \rightarrow p)$
\end{itemize}

\footnotetext{
    \uline{模糊逻辑}:模糊逻辑常用于人工只能,其命题的真值在0和1之间,表示不同的程度。
    \begin{itemize}
        \item 模糊逻辑中命题的反是1减命题的真值。
        \item 模糊逻辑中命题的交是所有命题真值的最小值。
        \item 模糊逻辑中命题的并是所有命题真值的最大值。
    \end{itemize} 
}

\subsection{证明}
\subsubsection*{证明方法和技巧}
\begin{itemize}
    \item 直接证明、间接证明(反证法)
    \item 利用矛盾证明(归谬证明)、利用等价证明
    \item 存在性证明、唯一性证明
    \item 穷举证明、数学归纳法、结构归纳法
    \item 康托尔对角化方法、组合证明
\end{itemize}

\subsubsection*{开放问题}
\begin{itemize}
    \item 费马大定理:$n > 2, \lnot \exists x,y,z \in \mathbb{Z}, that \ x^n + y^n = z^n$.(费马大定理已证明)
    \item $3x+1$猜想:设$T(x) = \begin{cases} \dfrac x2 & \text{x is even} \\ 3x+1 & \text{x is odd} \end{cases}$,对于所有的正整数,重复使用$T(x)$,最终结果必为1.
\end{itemize}

\section{集合、函数、数列}

\subsection{集合}
\begin{itemize}
    \item $\mathbb{N}$:自然数 \quad $\mathbb{Z}$:整数 \quad $\mathbb{Z^+}$:正整数 \quad $\mathbb{Q}$:有理数 \quad $\mathbb{R}$:实数
    \item 集合是无序的。
    \item element $\in$ set, subset $\subseteq$ set.\, $\emptyset \subseteq S, S \subseteq S$
    \item 集合的基数指集合中的不同元素的个数,记作$\lvert S \rvert$。若$\lvert S \rvert \to \infty$,则集合$S$为无限集合,否则为有限集合。
    \item 集合的幂:集合$S$的幂为$S$所有子集的集合,记作$P(S)$。\\ 如$P(\{1,2\}) = \{\emptyset, \{1\}, \{2\}, \{1,2\}\}$.
    \item 有序$n$元组:有序$n$元组$(a_1, a_2, \cdots, a_n)$是有顺序的。当$n=2$时,称为有序二元组。
    \item 笛卡儿积:$A \times B = \{(a,b) \mid a \in A \land b \in B\}$ \\ $A_1 \times A_2 \times \cdots \times A_n = \{(a_1, a_2,\cdots, a_n) \mid a_i \in A_i \}$
    \item 集合基本运算:$A \cap B$ \quad $A \cup B$ \quad $\displaystyle \bigcap _{i=1}^n A_i$ \quad $\displaystyle \bigcup _{i=1}^n A_i$ \quad $A - B$ \quad $\overline A$.
    \item 一个有限集合或基数等于正整数集的集合,称为可数集合;否则称为不可数集合。(无限集合可能是可数的,只要基数等于正整数集即可。)\\ 正整数集的基数记为$\aleph_0$,即$\lvert \mathbb{Z^+} \rvert = \aleph_0$。(aleph null,阿列夫零)
\end{itemize}

\subsubsection*{常用集合等式}
\begin{itemize}
    \item $A \cup (B \cup C) = (A \cup B) \cup C$ \quad $A \cap (B \cap C) = (A \cap B) \cap C$
    \item $A \cap (B \cup C) = (A \cap B) \cup (A \cap C)$ \quad $A \cup (B \cap C) = (A \cup B) \cap (A \cup C)$
    \item $\overline{A \cup B} = \overline A \cap \overline B$ \quad $\overline{A \cap B} = \overline A \cup \overline B$
    \item $A \cup \overline A = U$ \quad $A \cap \overline A = \emptyset$
\end{itemize}

\footnotetext{
    \uline{模糊集合}:全集$U$中的每个元素在集合$S$中都有成员度,处于0和1之间。
    \begin{itemize}
        \item 模糊集合的补集的元素的成员度为1减去该元素在原集合中的成员度。
        \item 模糊集合的并集是取元素在各集合中成员度的最大值。
        \item 模糊集合的交集是取元素在各集合中成员度的最小值。
    \end{itemize}
}

\subsection{函数}
\begin{itemize}
    \item 若$f_1, f_2$均是从$A$到$\mathbb{R}$的函数,则$f_1 + f_2, f_1 f_2$也是从$A$到$\mathbb{R}$的函数,且\[ (f_1 + f_2)(x) = f_1(x) + f_2(x) \] \[ (f_1 f_2)(x) = f_1(x) f_2(x) \]
    \item 若$f$是从$A$到$B$的函数,$S \subset A$,则$f(S) = \{f(s) \mid s \in S\}$。
    \item 若$f$是从$A$到$B$的函数,对于集合$B$中的任何一个元素$b$,都有且仅有一个属于集合$A$的元素与之对应,使得$f(a) = b$,则称$f$为一一对应或双射。
    \item 若$f$是从$A$到$B$的一一对应的函数,则存在$f$的反函数$f^{-1}$。
    \item 若$g$是从$A$到$B$的函数,$f$是从$B$到$C$的函数,则记\[ f \circ g = f(g(x))\]
    \item floor function: $\lfloor x \rfloor \text{ or } [x]$ \quad ceiling function: $\lceil x \rceil$
\end{itemize}

\subsection{数列}
\begin{itemize}
    \item 斐波那契数列:$f_0 = 0, f_1 = 1, f_n = f_{n-1} + f_{n-1}$ \\
    一对刚出生的兔子放在一个食物充足的孤岛上,假设兔子2个月才能长大成熟,每个月每一对成熟的兔子会繁殖出一对新的兔子,并且兔子不会死去,兔子的对数满足斐波那契数列。
    \item 几何数列$ar^k$的前$n$项和:$S = \dfrac{a(r^n-1)}{r-1}$.
    \item $\displaystyle \sum _{k=1}^n k = \dfrac{n(n+1)}{2}$
    \item $\displaystyle \sum _{k=1}^n k^2 = \dfrac{n(n+1)(2n+1)}{6}$
    \item $\displaystyle \sum _{k=1}^n k^3 = \dfrac{n^2 (n+1)^2}{4}$
    \item $\displaystyle \sum _{k=0}^\infty x^k (\lvert x \rvert < 1) = \dfrac{1}{1-x}$
    \item $\displaystyle \sum _{k=1}^\infty kx^{k-1} (\lvert x \rvert < 1) = \dfrac{1}{(1-x)^2}$
\end{itemize}

\section{算法、整数、矩阵}

\subsection{算法}
\begin{itemize}
    \item 常见算法:顺序搜索、二分搜索、冒泡排序、插入排序 \ldots
    \item 贪心算法:每一步都寻找局部最优解的算法,最后的结果不一定是最优。例如凑硬币的问题使用贪心算法,但是最后所得解即为最优解。
\end{itemize}

\subsubsection*{停机问题}
停机问题:是否存在一个程序,它的输入是某个计算机程序,输出是这个计算机程序能否正常结束。\uline{这样的程序是不存在的}。
\begin{proof}
构造一个程序$H(P,I)$,其中$P$为一个计算机程序,$I$为$P$的输入;其输出为$P$能否正常结束,当$P$停机时,输出字符串“停机”,当$P$死循环时,输出字符串“死循环”。

因为计算机程序$P$的源代码同时也是文本,同样可以作为$P$的输入。假设另一程序$K(P)$,其运行结果与$H(P,P)$相反,当$H(P,P)$输出为“死循环”,那么$K(P)$停机,当$H(P,P)$输出为“停机”,那么$K(P)$死循环。

此时考虑$H(K,K)$,如果$H(K,K)$输出为“死循环”,也即$K(K)$死循环,但是由上述$K(P)$的功能,可知$K(P)$停机。两种结果相矛盾,$H(P,P)$不能得到正确的结果,因此不存在这样的程序。
\end{proof}

\subsubsection*{大$O$记法}
\begin{itemize}
    \item 存在常数$C, k$使得当$x > k$时,$|f(x)| \le C|g(x)|$,则称$f(x)$是$O(g(x))$.
    \item $f(x) = a_n x^n + a_{n-1} x^{n-1} + \cdots + a_1 x + a_0$,$f(x)$是$O(x^n)$.
    \item $1 + 2 + 3 + \cdots + n$是$O(n^2)$,$n!$是$O(n^n)$,$\log n$是$O(n)$,$\log(n!)$是$O(n \log n)$.
    \item 常用的有$1, \log n, n, n\log n, n^2, 2^n, n^n$.
    \item 若$f_1(x)$是$O(g_1(x))$,$f_2(x)$是$O(g_2(x))$,则$(f_1 + f_2)(x)$是$O(\max(|g_1(x)|,|g_2(x)|))$,$(f_1 f_2)(x)$是$O(g_1(x) g_2(x))$.
    \item 存在常数$C, k$使得当$x > k$时,$|f(x)| \ge C|g(x)|$,则称$f(x)$是$\Omega (g(x))$.
    \item 若$f(x)$是$O(g(x))$且$f(x)$是$\Omega (g(x))$,则$f(x)$是$\Theta (g(x))$,或称$f(x)$是$g(x)$阶的。
    \item $f(x) = a_n x^n + a_{n-1} x^{n-1} + \cdots + a_1 x + a_0$,$f(x)$是$x^n$阶的。
\end{itemize}

\subsubsection*{算法复杂度}
\begin{itemize}
    \item 常数复杂度:$\Theta(1)$ \quad 对数复杂度$\Theta(\log n)$ \quad 线性复杂度$\Theta(n)$ \\ 多项式复杂度$\Theta(n^a)$ \quad 指数复杂度$\Theta(b^n)$ \quad 阶乘复杂度$\Theta(n!)$
    \item 最坏情形复杂度、平均情形复杂度
    \item NP类问题:在多项式复杂度内无法解决,但是可在多项式复杂度内验证答案是否正确。\\ P类问题:易解问题,可在多项式复杂度内解决的。
\end{itemize}

\subsection{整数}
\subsubsection*{整除与余数}

$a$整除$b$:记作$a | b$,也即$b / a$为整数。$a = dq + r$,则$q = a \ \mathrm{div} \ d, r = a \bmod d$.
\begin{itemize}
    \item 若$a|b, a|c$,则$a|(mb+nc)$.
    \item 若$a|b$,则$\forall c \in \mathbb{Z},a|bc$.
    \item 若$a|b,b|c$,则$a|c$.
\end{itemize}

若$m$能整除$a - b$,则$a \bmod m = b \bmod m$,此时$a,b$ \uline{同余},记作$a \equiv b \pmod m$,且存在整数$k$使得$a = b + km$.
\begin{itemize}
    \item 若$a \equiv b \pmod m, c \equiv d \pmod m$,则$a+c \equiv b+d \pmod m, ac = bd \pmod m$.
    \item $(a + b) \bmod m = ((a \bmod m)+(b \bmod m)) \bmod m$ \\ $ab \bmod m = ((a \bmod m)(b \bmod m)) \bmod m$
\end{itemize}

同余的应用:
\begin{itemize}
    \item 哈希函数:$h(k) = k \bmod m$,其中$k$为记录的key。哈希值与$key$并不是一一对应的。
    \item 线性同余法生成伪随机数:选定4个整数$a,c,m,x_0,2 \le a < m, 0 \le c < m, 0 \le x_0 < m$,其中$x_0$称为种子。使用$x_{n+1} = (a x_n + c) \pmod m$生成一个伪随机数序列。
    \item 数字校验中的奇偶校验:对于数字序列$x_1 x_2 \cdots x_n$,奇偶校验位$x_{n+1} = (x_1 + x_2 + \cdots + x_n) \bmod 2$.\\ 其它的如通用产品代码UPC,也即商品包装上的条形码;国际标准书号ISBN等。
\end{itemize}

\subsubsection*{素数}
\begin{itemize}
    \item 算数基本定理:任何大于1的正整数,都可以唯一地写成两个或多个素数的乘积。
    \item 若$n$为合数,则$n$必有小于或等于$\sqrt n$的素因子。
    \item 有无限多的素数。
    \item 梅森素数:若$p$为素数,且$2^p - 1$也为素数,则称$2^p - 1$为梅森素数。
    \item 素数定理:当$x$无限增长时,不超过$x$的素数的个数与$x / \ln x$之比趋向于1。
    \item 哥德巴赫猜想:任何一个大于2的奇数,必为2个素数之和。
    \item 孪生素数猜想:有无限多的孪生素数。
\end{itemize}

\subsubsection*{最大公约数和最小公倍数}
最大公约数$\gcd(a,b)$,最小公倍数$\mathrm{lcm} (a, b)$.
\begin{itemize}
    \item $ab = \gcd (a,b) \cdot \mathrm{lcm} (a,b)$
    \item 使用欧几里得算法求最大公约数:若$a,b,q,r \in \mathbb Z, a = dq +r$,则$\gcd(a,b) = \gcd(b,r)$,辗转相除,求得最大公约数。
    \item 贝祖等式:若$a, b \in \mathbb Z^+$,则$\exists s, t \in \mathbb Z, gcd(a,b) = sa + tb$($s,t$可能不全为正)。
    \item 若$a, b, c \in \mathbb Z^+, \gcd(a,b) = 1, a|bc$,则$a|c$.
    \item $p$为素数,$p|a_1a_2 \cdots a_n$,则$\exists a_i \in \{a_1, a_2, \cdots ,a_n\} ,p|a_i$.
    \item 若$m \in \mathbb Z^+, a,b,c \in \mathbb Z, ac \equiv bc \pmod m, \gcd(c,m) = 1$,则$a \equiv b \pmod m$.
\end{itemize}

\subsubsection*{相关问题}
\begin{itemize}
    \item 线性同余方程:$ax \equiv b \pmod m$,解为$x_0 + k \frac md, d = \gcd (a,n)$. \\ 解法:利用$a \overline a \equiv 1 \pmod m$,其中$\overline a$是$a$的模$m$的逆,方程两侧同乘$\overline a$,即可解得。

    \item 若$a, m \in \mathbb Z$互素,$m > 1$,则存在$a$的模$m$的逆,且该逆模$m$是唯一的。\\ 例如:$a = 8, m = 3$,则$\overline a = 2 + 3k , \overline a \bmod 3 = 2$.

    \item 中国剩余定理:若$m_1, m_2, \cdots, m_n$为两两互素的素数,则\[x \equiv a_1 \pmod {m_1} \]\[ x \equiv a_2 \pmod {m_2} \]\[ \cdots \]\[ x \equiv a_n \pmod {m_n}\]有唯一的模$m$的解,其中$m = m_1 m_2 \cdots m_n$。可以使用此定理进行大整数的运算。

    例如:取$m$为99,98,97,95,123684可以表示为(33,8,9,89),413456可以表示为(32,92,42,16),则123684+413456可以使用如下计算:
    \begin{align*}
        (33,8,9,89) + & (32,92,42,16) \\ &= (65 \bmod 99, 100 \bmod 98, 51 \bmod 97, 105 \bmod 95) \\ &= (65,2,51,10)
    \end{align*}
    
    根据中国剩余定理,
    \begin{align*}
        x &= 65 \pmod {99} \\ x &= 2 \pmod {98} \\
        x &= 51 \pmod {97} \\ x &= 10 \pmod {95}
    \end{align*}
    可得537140是小于$99 \times 98 \times 97 \times 95$的惟一解。因此,可以使用中国剩余定理进行大整数运算。

    \item 费马小定理:$p$为素数,$a$是不能被$p$整除的整数,则$a^{p-1} \equiv 1 \pmod p$. 因而对于所有的整数$a$,都有$a^p \equiv p \pmod p$.

    \item 伪素数:$b$为正整数,$n$为正合数,且$b^{n-1} \equiv 1 \pmod n$,则$n$称为基数为$b$的伪素数。

    \item 卡米切尔数:$b$为正整数,$n$为正合数,且$\gcd (b,n) = 1, b^{n-1} \equiv 1 \pmod n$,则称$n$卡米切尔数。

    \item 原根:整数$r$和素数$p$,若$1 < r < p, 0 < i < p$,且$r^i \bmod p$两两不同(有$p-1$种结果),则称$r$为$p$的一个原根。也即$[1,p-1]$上的所有整数$i$,当且仅当$i = p-1$时,$r^i \equiv 1 \pmod p$.
\end{itemize}

\subsubsection*{密码}
\begin{itemize}
    \item 密码系统:包含5部分,明文$\mathcal P$、密文$\mathcal C$、密钥空间$\mathcal K$、加密函数$\mathcal E$、解密函数$\mathcal D$。若密钥$k$对应的加密函数为$E_k$,揭秘函数为$D_k$,则$D_k(E_k(p)) = p$.

    \item 经典密码学:
    \begin{itemize}
        \item 凯撒密码

        \item 分组密码:将文本分成一定长度的单元,每个单元按照一定的方式加密。
    
        \item 转置密码:设函数$\sigma([1,2,\cdots,n]) = [m_1,m_2, \cdots, m_n]$,将文本分成长度为$n$的分组,按$1,2,\cdots, n$的位置,分别移位$m_1,m_2,\cdots,m_n$。
    \end{itemize}

    \item 公钥密码学:加密密钥是公开的,解密密钥是保密的。

    例如,RSA密码系统:加密需要两个整数$n,e$,其中公钥$n = pq$为两个大素数的乘积,$p,q$各约200位,$e$为与$(p-1)(q-1)$互素的指数。加密过程首先将文本翻译成整数,然后分组,每一组为一个大整数$M$;密文;$C = M^e \bmod n$.

    解密需要解密密钥$d$,$d$为$e$模$(p-1)(q-1)$的逆。也即$de \equiv 1 \pmod {(p-1)(q-1)} \to de = 1 + k(p-1)(q-1)$.

    因此$C^d \equiv (M^e)^d \equiv M^{de} \equiv M^{1+k(p-1)(q-1)} \pmod n$。假定$\gcd (M, p) = \gcd (M,q) = 1$(这一关系不成立的可能性很小),则根据费马小定理
    
    $C^d \equiv M \cdot (M^{p-1})^{k(q-1)} \equiv  M \cdot 1 \equiv M \pmod p \\ C^d \equiv M \cdot (M^{q-1})^{k(p-1)} \equiv  M \cdot 1 \equiv M \pmod q$.

    根据中国剩余定理,$C^d \equiv M \pmod {pq}$.
\end{itemize}

\subsection{矩阵}
对于矩阵$\bf A, \bf B$,
\begin{itemize}
    \item $\bf A + \bf B = [a_{ij} + b_{ij}]$

    \item $\bf A \bf B = [c_{ij}], c_{ij} = a_{i1} b_{1j} + a_{i2} b_{2j} + \cdots + a_{ik} b_{kj}$

    \item $n$阶单位阵$\bf I_n$,矩阵的幂 $\bf A^0 = \bf I_n \quad \bf A^r = \underbrace{\bf A \bf A \cdots \bf A}_{\text{r次}}$

    \item 转置$\bf A^t$,对称阵$\bf A^t = \bf A$

    \item 0-1矩阵,元素只有0和1的矩阵。对于0-1矩阵:
    \begin{itemize}
        \item $\bf A \land \bf B = [a_{ij} \land b_{ij}]$,$\bf A \lor \bf B = [a_{ij} \lor b_{ij}]$

        \item 布尔积 $\bf A \odot \bf B = [c_{ij}],c_{ij} = (a_{i1} \land b_{1j}) \lor (a_{i2} \land b_{2j}) \lor \cdots \lor (a_{ik} \land b_{kj})$

        \item 布尔幂$\bf A^{[r]} = \underbrace{\bf A \odot \bf A \odot \bf A \odot \cdots \odot \bf A}_{\text{r次}}$
    \end{itemize}
\end{itemize}